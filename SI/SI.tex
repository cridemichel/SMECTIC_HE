%% ****** Start of file aiptemplate.tex ****** %
%%
%%   This file is part of the files in the distribution of AIP substyles for REVTeX4.
%%   Version 4.1 of 9 October 2009.
%%
%
% This is a template for producing documents for use with 
% the REVTEX 4.1 document class and the AIP substyles.
% 
% Copy this file to another name and then work on that file.
% That way, you always have this original template file to use.

\documentclass[aip,graphicx]{revtex4-1}
%\documentclass[aip,reprint]{revtex4-1}
\usepackage{amsmath}
\usepackage{graphicx}
\usepackage{xcolor}
\renewcommand{\thefigure}{S\arabic{figure}}
%\usepackage{siunitx}

\draft % marks overfull lines with a black rule on the right

\begin{document}

% Use the \preprint command to place your local institutional report number 
% on the title page in preprint mode.
% Multiple \preprint commands are allowed.
%\preprint{}

\title{Magnetic-field Assisted Assembly of Colloidal Ellipsoids}

% repeat the \author .. \affiliation  etc. as needed
% \email, \thanks, \homepage, \altaffiliation all apply to the current author.
% Explanatory text should go in the []'s, 
% actual e-mail address or url should go in the {}'s for \email and \homepage.
% Please use the appropriate macro for the type of information

% \affiliation command applies to all authors since the last \affiliation command. 
% The \affiliation command should follow the other information.

%\author{}

\author{Antara Pal}
\email{antara.pal@fkem1.lu.se}
\affiliation{Division of Physical Chemistry, Department of Chemistry, Lund University, Lund, Sweden}
\author{Thiago H.Ito}
\affiliation{Division of Physical Chemistry, Department of Chemistry, Lund University, Lund, Sweden}
\author{Md. Arif Kamal}
\affiliation{Centre Interdisciplinaire de Nanoscience de Marseille (CINaM), CNRS, Aix Marseille University, Marseille, France}
\altaffiliation[Current Address: ]{Division of Physical Chemistry, Department of Chemistry, Lund University, Lund, Sweden}
\author{Andrei V. Petukhov}
\affiliation{Van't Hoff Laboratory for Physical and Colloid Chemistry, Utrecht University, The Netherlands}
\author{Peter Schurtenberger}
\email{peter.schurtenberger@fkem1.lu.se}
\affiliation{Division of Physical Chemistry, Department of Chemistry, Lund University, Lund, Sweden}


% Collaboration name, if desired (requires use of superscriptaddress option in \documentclass). 
% \noaffiliation is required (may also be used with the \author command).
%\collaboration{}
%\noaffiliation

%\date{\today}

%\pacs{}% insert suggested PACS numbers in braces on next line

\maketitle {}%\maketitle must follow title, authors, abstract and \pacs

\begin{figure}[]
\centering
\includegraphics [width=1\linewidth]{paperclip.png}
\caption{(a) Real space image of a classical smectic phase formed by colloidal rods and (b) expected x-ray diffraction pattern for it. (c) Real space image of \textit{oblate} smectic phase formed by ellipsoids and (d) expected diffraction x-ray diffraction pattern for it. (e) Smectic phase with layer fluctuation and (d) the corresponding change in the diffraction pattern.}
\label{paperclip}
\end{figure}

%Fig.\ref{paperclip}(a) shows the real space structure of a classical smectic phase consists of rod-like colloids where the particles are aligned along their long axes. As a result, the smectic layers are along their length. Fig.\ref{paperclip}(b) shows the corresponding Fourier space image or the expected x-ray diffraction pattern~\cite{kuijk2012phase, byelov2013situ}. Fig.\ref{paperclip}(c) represents the smectic phase formed by our ellipsoidal particles in the presence of an external field. One can observe that the particles are aligned with their short axes being parallel to the external field. The red doubled arrows show the spacings along the layers. There are also correlations between particles belong to different layers as shown by green lines which result in the formation of a diffused scattering line as shown in Fig.\ref{paperclip}(d) in green. Although from our drawing it seems that the long axes are also aligned, it is not true. In reality they can still rotate around the field as has been shown in Fig. 2. For the drawing, we have not considered all possible rotational conformations but only one; otherwise it will be extremely difficult to draw and point out the different spacings while including the all rotational conformations. Further, The layers in the smectic phase are not rigid but can fluctuate as can be seen in Fig.\ref{paperclip}(e) which elongated the smectic peak in vertical direction as shown by dark yellow in Fig.\ref{paperclip}(f). Combining all the aforementioned effects it is possible to understand the paperclip shape for the smectic phase in our study.

The real space structure of a classical smectic phase consisting of rod-like colloids is schematically shown in Fig.\ref{paperclip}(a). In this case the particles align along their long axes. As a result, the smectic layers form in a direction parallel to the length of the rods. The corresponding Fourier space image or the expected x-ray diffraction pattern of the aforementioned smectic phase is shown in Fig.\ref{paperclip}(b)~\cite{kuijk2012phase, byelov2013situ}. However, the smectic phase formed by ellipsoidal particles which in the presence of an external field align with their short axes parallel to the external field, has the appearance as shown in Fig.\ref{paperclip}(c). The double headed red arrows indicate the spacings along the smectic layers. Correlations between particles which belong to different layers (as indicated by green lines) results in the formation of a diffused scattering line as shown in Fig.\ref{paperclip}(d) in green. Although our schematic gives an impression that the long axes are also aligned, but this is not the case in general. As mentioned before the particles align with their short axes along the field direction and their long axes are free to rotate about this direction (Fig. 2). For the sake of clarity in illustration, we have chosen to highlight only one such possible conformation out the ensemble of all possible rotational conformations. Further it is important to note that the smectic layers in this case are not rigid but can fluctuate, Fig.\ref{paperclip}(e), resulting thereby in an elongation of the smectic peak in vertical direction as indicated by dark yellow in Fig.\ref{paperclip}(f).

\bibliography{ref_dubble}


\end{document}